\documentclass[letterpaper,10pt]{article}
\usepackage[utf8]{inputenc}
\usepackage{xifthen}
\usepackage[T1]{fontenc}
\usepackage[dvipsnames]{xcolor}
\usepackage[colorlinks=true,urlcolor=blue]{hyperref}
\usepackage{titlesec}
\usepackage[margin=1in]{geometry}
\usepackage{longtable}
\usepackage{titling}

\author{Matthew Critchlow}

\date{\today}
\renewcommand{\maketitle}{
\par{\centering{\Huge  \textsc{\theauthor}}\par}
{\footnotesize\hfill{}\color{lightgray}(Last updated \thedate.)}}

%Setting the font I want:
\renewcommand{\familydefault}{\sfdefault}
\renewcommand{\sfdefault}{ppl}

\newcommand{\entry}[4]{
\ifthenelse{\isempty{#3}}
{\slimentry{#1}{#2}}{

\begin{minipage}[t]{.15\textwidth}
\hfill \textsc{#1}
\end{minipage}
\hfill\vline\hfill
\begin{minipage}[t]{.80\textwidth}
{\bf#2}---\textit{#3}. \footnotesize{#4}
\end{minipage}\\
\vspace{.25cm}
}}

\newcommand{\slimentry}[2]{

\begin{minipage}[t]{.15\textwidth}
\hfill \textsc{#1}
\end{minipage}
\hfill\vline\hfill
\begin{minipage}[t]{.80\textwidth}
#2
\end{minipage}\\
\vspace{.25cm}
}

%Instition macros, because laziness
\newcommand{\uci}{University of California, Irvine}
\newcommand{\ucsd}{University of California, San Diego}
\newcommand{\ucsb}{University of California, Santa Barbara}
\newcommand{\usd}{University San Diego}

%Section spacing and format:
\titleformat{\section}{\Large\scshape\raggedright}{}{1em}{}[\titlerule]
\titlespacing{\section}{0pt}{3pt}{7pt}
\titleformat{\subsection}{\large\sc\centering}{}{0em}{\underline}%[\rule{3cm}{.2pt}]
\titlespacing{\subsection}{0pt}{7pt}{7pt}

\setlength{\parindent}{0in}
\setlength{\parindent}{0in}

\begin{document}

\maketitle

\section{Basic Info}

\vspace{.25cm}

\begin{minipage}[t]{.5\linewidth}

\begin{tabular}{rp{.75\linewidth}}
    \textsc{Email:}     & \href{mailto:matt.critchlow@gmail.com}{matt.critchlow@gmail.com}\\
    \textsc{www:}&\href{https://mcritchlow.github.io}{mcritchlow.github.io}
\end{tabular}
\end{minipage}
\begin{minipage}[t]{.5\linewidth}
\begin{tabular}{rl}
\textsc{Github:} & \href{http://github.com/mcritchlow}{mcritchlow}\\
\textsc{PGP:}&\href{http://pgp.mit.edu/pks/lookup?op=get&search=0xE53B4D1FE0B14937}{E53B4D1FE0B14937}
\end{tabular}
\end{minipage}

\vspace{.25cm}

\section{Experience}

\entry{2017}{Lead Application Developer}{\ucsd}{

Collaborate directly with the Lead DevOps engineer to introduce several changes to our configuration management,
continous integration and delivery strategies through the use of Ansible, Jenkins, and possibly orchestrated containers. Lead open source community engagement and technical initiatives that are
of value to {\ucsd} with the Samvera, Blacklight, and Fedora projects. Work with entire development team to successfully
deliver a new Digital Asset Management System based on Fedora and Samvera.
}
\entry{2013--2017}{Manager of Development and Web Services}{\ucsd}{

In this position I was able to successfully advocate and oversee the team's transition to adopting an open source
Library repository framework, then called Hydra. This transition included an entirely new workflow and set of
technologies, which I introduced successfully and the entire Library organization realized substantial value from. This
included the transition to bi-weekly development iterations (Sprints), requiring comprehensive test coverage for our
software applications, implementing a continous integration workflow and a chatops deployment strategy.
}
\entry{2009--2013}{Web Technical Manager}{\ucsd}{

Served as manager and technical expert for the Web Services team. Collaborated with a key Library stakeholder to
introduce Agile principles for the first time in a Library project. This was done for the creation of the Library mobile
website which was very well received. Other projects included a complete overhaul of the Library Public Website, the
creation of a new Library Intranet, and several other critical projects such as the San Diego Technology Archive.}
\entry{2006--2009}{Programmer/Analyst}{\ucsd}{

In this position I wrote a Audio Reserves application that students used
to access playlists of content that was required listening for their course. This was a high demand application on
campus. The software was written in Java and JSP, but a triplestore was used instead of a traditional relational
database. This made the CRUD interactions for the application more complex, but allowed for very robust metadata. I also
wrote the initial editing support for the Digital Asset Management System triplestore repository that interacted with
Apache Jena to manipulate the RDF graphs for our digital objects. }

\entry{2004--2006}{Programmer/Analyst}{\uci}{

I worked in the Registrar's Office software development team. I was
responsible for writing software that students used to look up time-sensitive information such as their grades and
course schedules. The software was predominately written in Java using the Apache Struts framework. I also learne a lot of Perl on the job, as
the lead developer at the time thought it was the greatest thing man has ever invented.}

\section{Workshops}

\entry{2017}{\href{https://ucsdlib.github.io/git-novice/}{``Software Carpentry: Version Control with Git''}}{Samvera Connect}{Taught the Version Control with Git
course of the
Software Carpentry curriculum. Rewritten completely with Alex Dunn and Chrissy Rissmeyer from UC Santa Barbara to cater
to an audience of metadata librarians.}

\entry{2017}{\href{https://ucsdlib.github.io/git-novice/}{``Software Carpentry: Version Control with Git''}}{{\ucsb}}{Taught the Version Control with Git course of the
Software Carpentry curriculum. Rewritten completely with Alex Dunn and Chrissy Rissmeyer from UC Santa Barbara to cater
to an audience of metadata librarians.}

\entry{2016}{\href{https://ucsdlib.github.io/workshops/posts/upcoming/swc/software-carpentry-sio/}{``Software Carpentry''}}{Scripps Institute of Oceanography, {\ucsd}}{Taught the Version Control with Git
course of the Software Carpentry curriculum.
Substantially rewritten by me, for a grad-student focus after initial feedback. TA for The Unix Shell and Programming with Python courses.}

\entry{2016}{\href{https://ucsdlib.github.io/workshops/posts/library-carpentry/past/library-carpentry/}{``Library Carpentry''}}{The Library, {\ucsd}}{Taught the Version Control with Git course of the Library Carpentry
curriculum. TA for Unix Shell and Open Refine courses.}

\entry{2016}{\href{https://ucsdlib.github.io/workshops/posts/posts/software-carpentry/}{``Software Carpentry''}}{Biomedical Library, {\ucsd}}{Taught the Version Control with Git
course of the Software Carpentry curriculum. TA for The Unix Shell and Programming with Python courses.}

\section{Presentations}
\entry{2017}{``Code Readability and Productivity Tips''}{Campus LISA Conference, {\ucsd}}{Paper
presenting common software development strategies and best practices. This covered topics such as refactoring, using
good names, patterns developers should be able to recognize and name, as well as productivity tips in both Ruby and .NET
languages.}

\entry{2016}{``Open Source at UC San Diego''}{Campus LISA Conference, {\ucsd}}{Paper
presenting how the UC San Diego Library has been able to participate in open source development, and the challenges
presented by working with upstream projects using the Apache 2 license, and how we worked around it.}

\entry{2015}{``Co-working Space: Online Tools for Collaboration''}{Campus LISA Conference, {\ucsd}}{Paper
presenting current tools that can be used for online collaboration and discussion of how they were integration into our
workflows to make life more efficient. Covered technologies like Slack, Github, and Confluence.}

\entry{2014}{\href{https://www.slideshare.net/mattcritchlow/sca2014-ucsdfinal}{``Programmer and Archivist
Collaboration''}}{Society of California Archivists}{Forum-format presentation detailing how I worked closely with the Digital Library Project Manager, Cristela Garcia-Spitz, to come up with innovative ways of tracking progress of Digital Library project as well as development
Sprints}

\entry{2014}{\href{https://www.slideshare.net/mattcritchlow/the-evolution-of-the-uc-san-diego-library-dams}{``The
Evolution of the UC San Diego Library DAMS''}}{Digital Initiatives Symposium, {\usd}}{Paper
presenting an over of the {\ucsd} DAMS, the history behind it, and the use cases that drove the creation of a new system
to support them}

\entry{2014}{\href{http://www.slideshare.net/mattcritchlow/software-development-process-36844544}{``Software Development
Process''}}{Campus LISA Conference, {\ucsd}}{Paper
presenting the test driven development process and related tooling I helped implement}

\entry{2014}{\href{https://www.slideshare.net/mattcritchlow/uc-san-diego-campus-lisa-2014-source-code}{``Source Code
Management''}}{Campus LISA Conference, {\ucsd}}{Paper presenting the migration to using git, GitHub, and the git flow pattern for local development and release
management that I helped implement.}

\entry{2013}{\href{http://www.slideshare.net/mattcritchlow/c4-l-alltehmetadatas2013final}{``ALL TEH METADATAS
Re-revisited''}}{Code4Lib Conference}{Paper
presenting a comprehensive ontology modeling project I was a primary contributor to at {\ucsd}}


\entry{2013}{\href{http://www.slideshare.net/mattcritchlow/ucsd-library-hot-topics-webinars-part-2-final}{``Metadata and
Repository Services for Research Data Collections''}}{Duraspace Hot Topics Webinar Series}{Part of a three part webinar series in which I spoke about our Digital
Asset Management System and related metadata modeling to support Research Data Collections at {\ucsd}}

\section{Publications}

\entry{2010}{\href{http://journal.code4lib.org/articles/4642}{``Using an Agile-based Approach to Develop A Library
Mobile Website''}
}{Matt Critchlow, Lia Friedman, Daniel Suchy}{Code4Lib Journal}

\entry{2003}{``An Environment for Managing Evolving Product Line Architectures''
}{Ping Chen, Matt Critchlow, Akash Garg, Chris Van der Westhuizen, and André van der Hoek}{In Proceedings of the International Conference on Software Maintenance, Amsterdam, Netherlands, September 2003}

\entry{2003}{``Differencing and Merging within an Evolving Product Line Architecture''
}{Ping Chen, Matt Critchlow, Akash Garg, Chris Van der Westhuizen, and André van der Hoek}{In Proceedings of the Fifth International Workshop on Product Family Engineering, Siena, Italy, November 2003}

\section{Institutions}

\entry{2011--2013}{Professional Certificate in Project Management}{\ucsd}{}
\entry{1999--2004}{B.S. in Computer Science}{\uci}{}

\section{Technical}
\entry{Languages}{Ruby, Java, PHP, Python, JavaScript, Bash, CSS, HTML.}{}{}
\entry{Markup}{Markdown, YAML, HAML, {\LaTeX}}{}{}

\section{Hobbies}
\entry{Open Source}{\href{https://solus-project.com/}{Solus (Linux Distibution)},
{\href{https://github.com/samvera}{Samvera}}}{}{}
\entry{Other}{Cooking, Playing Guitar, Travel, Surfing}{}{}

\end{document}
